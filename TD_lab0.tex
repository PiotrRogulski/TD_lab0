\documentclass[a4paper,12pt,notitlepage]{article}

\usepackage{amsmath, amssymb, amsthm}
\usepackage{mathtools}
\usepackage[math]{fontspec}
\usepackage[nosingleletter, lastparline]{impnattypo}
\usepackage[polish]{babel}
\usepackage{bookmark}
\usepackage[margin=1in]{geometry}
\usepackage[babel=true, tracking=true]{microtype}

\linespread{1.3}
\setlength{\parindent}{0pt}

\title{\textbf{TD -- laboratoria 0}}
\author{Piotr Rogulski 305867 \\ Szymon Sieradzki 305881}
\date{\today}

\begin{document}

\maketitle
\section{Konfiguracja ipv4}
\subsection{Konfiguracja routera R1}
\subsection{Konfiguracja routera R2}
Po połączeniu z routerem R2, ipv4 address routera nie jest ustawiony.
\\ Obrazek z logów \\
By ustawić ipv4 wchodzimy w tryb konfiguracji i ustawiamy ipv4 adres.
\\Obrazek logów \\
Adres ipv4 routera R2 zostaje ustawiony na 192.168.0.2 z maską 255.255.255.0.
\\Obrazek logów\\
Pozostaje nam jeszcze ustawić \textit{line protocol} na \textit{up} przy pomocy \textit{no shutdown}. W tym celu ponownie otwieramy tryb konfiguracji.
\\Obrazek logów\\
R2 jest gotowy do łączenia z innymi routerami.
\\Obrazek logów\\
\subsection{Połączenie między routerami}
\section{Konfiguracja ipv6}
\subsection{Konfiguracja routera R1}
\subsection{Konfiguracja routera R2}
By skonfigurować R2 pod użycie ipv6 wchodzimy w tryb konfiguracji i ustawiamy ipv6 adres.
\\Obrazek logów \\
R2 znajduje się pod adresem 2001:DB8:0:1::1.
\\Obrazek logów\\\
Usuwamy ustawiony wcześniej ipv4 adres.
\\Obrazek logów\\
R2 jest gotowy do łączenia z innymi routerami przez ipv6.
\\Obrazek logów\\
\subsection{Połączenie między routerami}
\end{document}
