\documentclass[a4paper,12pt,notitlepage]{article}

\usepackage[svgnames]{xcolor}
\usepackage{amsmath, amssymb, amsthm}
\usepackage{mathtools}

\usepackage[math]{fontspec}
\usepackage[nosingleletter, lastparline]{impnattypo}
\usepackage[polish]{babel}
\usepackage{bookmark}
\usepackage[margin=1in]{geometry}
\usepackage[babel=true, tracking=true]{microtype}
\usepackage{shellesc}
\usepackage{minted}

\definecolor{bg}{rgb}{0.95,0.95,0.95}
\setminted{frame=single, bgcolor=bg, breaklines=true}
\setmintedinline{bgcolor=white}

\linespread{1.3}
\setlength{\parindent}{0pt}

\title{\textbf{TD -- laboratorium 0}}
\author{Piotr Rogulski 305867 \\ Szymon Sieradzki 305881}
\date{\today}

\IfFontExistsTF{JetBrainsMono-Regular}{
    \setmonofont{JetBrainsMono}[
        UprightFont = *-Light,
        BoldFont = *-Regular,
        ItalicFont = *-Light-Italic,
        Scale = MatchLowercase
    ]
}{}

\begin{document}

\maketitle

% \inputminted[label=Output, firstline=20]{text}{R1_1_copy.txt}

\section{Schemat adresacji}

\begin{table}[!htb]
    \begin{minipage}{.5\linewidth}
        \caption{Adresacja ipv4}
        \centering
        \begin{tabular}{c|c}
            \hline\hline
            R1 & \mintinline{text}{192.168.0.1/24} \\
            R2 & \mintinline{text}{192.168.0.2/24} \\
            \hline
        \end{tabular}
    \end{minipage}%
    \begin{minipage}{.5\linewidth}
        \caption{Adresacja ipv6}
        \centering
        \begin{tabular}{c|c}
            \hline\hline
            R1 & \mintinline{text}{2001:DB8:0:1::2/64} \\
            R2 & \mintinline{text}{2001:DB8:0:1::1/64} \\
            \hline
        \end{tabular}
    \end{minipage}
\end{table}

\section{Konfiguracja ipv4}

\subsection{Konfiguracja routera R1}

W celu konfiguracji routera R1, na początek należy nadać mu adres ipv4 \mintinline{text}{192.168.0.1/24}.%
\inputminted[label=Ustawianie adresu ipv4 dla R1, firstline=89, lastline=92]{text}{R1.txt}

Po ustawieniu adresu nie ma jeszcze komunikacji między urządzeniami w sieci, ponieważ protokół linii \mintinline{text}{Ethernet0/0} jest wyłączony.%
\inputminted[label=Włączanie protokołu linii, firstline=122, lastline=123]{text}{R1.txt}

Router R1 został pomyślnie skonfigurowany i może komunikować się z innymi urządzeniami w sieci poprzez protokół ipv4.%
\inputminted[label=Końcowa konfiguracja routera R1, firstline=129, lastline=154]{text}{R1.txt}

\subsection{Konfiguracja routera R2}

Po połączeniu z routerem R2, ipv4 address routera nie jest ustawiony.
\inputminted[label=Początkowa konfiguracja R2, firstline=88, lastline=91]{text}{R2.txt}
By ustawić ipv4 należy wejść w tryb konfiguracji i ustawić ipv4 adres.
\inputminted[label=Ustawianie ipv4 dla R2, firstline=113, lastline=120]{text}{R2.txt}
Adres ipv4 routera R2 zostaje ustawiony na 192.168.0.2 z maską 255.255.255.0.
\inputminted[label=R2 z ustawionym ipv4, firstline=121, lastline=124]{text}{R2.txt}
Pozostaje jeszcze ustawić \textit{line protocol} na \textit{up} przy pomocy \textit{no shutdown}. W tym celu należy otworzyć tryb konfiguracji.
\inputminted[label=Ustawianie line protocol na up, firstline=147, lastline=157]{text}{R2.txt}
R2 jest gotowy do łączenia z R1 przez Ethernet 0/0.
\inputminted[label=Końcowa konfiguracja R2 dla ipv4, firstline=162, lastline=185]{text}{R2.txt}

\subsection{Połączenie między routerami}

Powyższa konfiguracja pozwala na komunikację między routerami R1 i R2.%
\inputminted[label=Próba komunikacji R2 z R1, firstline=211, lastline=216]{text}{R2.txt}%
\inputminted[label=Próba komunikacji R1 z R2, firstline=155, lastline=160]{text}{R1.txt}

\section{Konfiguracja ipv6}

\subsection{Konfiguracja routera R1}

W celu skonfigurowania komunikacji przez protokół ipv6 należy wyłączyć protokół ipv4 oraz włączyć ipv6.%
\inputminted[label=Przełączanie na ipv6, firstline=171, lastline=177]{text}{R1.txt}

Należy także ustawić odpowiedni adres ipv6 dla routera R1.%
\inputminted[label=Ustawianie adresu ipv6 dla R1, firstline=237, lastline=238]{text}{R1.txt}

Router R1 został pomyślnie skonfigurowany i może komunikować się z innymi urządzeniami w sieci poprzez protokół ipv6.

\subsection{Konfiguracja routera R2}

By skonfigurować R2 pod użycie ipv6 należy wejść w tryb konfiguracji i ustawić ipv6 adres.
\inputminted[label=Ustawianie ipv6 dla R2, firstline=272, lastline=283]{text}{R2.txt}
R2 znajduje się pod adresem 2001:DB8:0:1::1.
\inputminted[label=R2 z ustawionym ipv6, firstline=309, lastline=313]{text}{R2.txt}
Teraz trzeba usunąć ustawiony wcześniej ipv4 adres i dodać unicast routing.
\inputminted[label=Usunięcie ipv4, firstline=324, lastline=329]{text}{R2.txt}
\inputminted[label=Dodanie unicast routing, firstline=376, lastline=382]{text}{R2.txt}
R2 jest gotowy do łączenia przez Ethernet 0/0 korzystając z ipv6.
\inputminted[label=Końcowa konfiguracja R2 dla ipv6, firstline=383, lastline=402]{text}{R2.txt}

\subsection{Połączenie między routerami}

Powyższa konfiguracja pozwala na komunikację między routerami R1 i R2.%
\inputminted[label=Próba komunikacji R2 z R1, firstline=416, lastline=421]{text}{R2.txt}%
\inputminted[label=Próba komunikacji R1 z R2, firstline=248, lastline=253]{text}{R1.txt}

\end{document}
