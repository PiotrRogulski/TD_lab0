\documentclass[a4paper,12pt,notitlepage]{article}

\usepackage[svgnames]{xcolor}
\usepackage{amsmath, amssymb, amsthm}
\usepackage{mathtools}

\usepackage[math]{fontspec}
\usepackage[nosingleletter, lastparline]{impnattypo}
\usepackage[polish]{babel}
\usepackage{bookmark}
\usepackage[margin=1in]{geometry}
\usepackage[babel=true, tracking=true]{microtype}
\usepackage{shellesc}
\usepackage{minted}

\definecolor{bg}{rgb}{0.95,0.95,0.95}
\setminted{numbers=left, frame=single, bgcolor=bg, breaklines=true}

\linespread{1.3}
\setlength{\parindent}{0pt}

\title{\textbf{TD -- laboratoria 0}}
\author{Piotr Rogulski 305867 \\ Szymon Sieradzki 305881}
\date{\today}

\begin{document}

\maketitle

% \inputminted[label=Output, firstline=20]{text}{R1_1_copy.txt}

\section{Konfiguracja ipv4}
\subsection{Konfiguracja routera R1}
\subsection{Konfiguracja routera R2}
Po połączeniu z routerem R2, ipv4 address routera nie jest ustawiony.
\inputminted[label=Output, firstline=88, lastline=111]{text}{R2.txt}
By ustawić ipv4 wchodzimy w tryb konfiguracji i ustawiamy ipv4 adres.
\inputminted[label=Output, firstline=113, lastline=120]{text}{R2.txt}
Adres ipv4 routera R2 zostaje ustawiony na 192.168.0.2 z maską 255.255.255.0.
\inputminted[label=Output, firstline=121, lastline=144]{text}{R2.txt}
Pozostaje nam jeszcze ustawić \textit{line protocol} na \textit{up} przy pomocy \textit{no shutdown}. W tym celu ponownie otwieramy tryb konfiguracji.
\inputminted[label=Output, firstline=147, lastline=157]{text}{R2.txt}
R2 jest gotowy do łączenia z R1 przez Ethernet 0/0.
\inputminted[label=Output, firstline=162, lastline=185]{text}{R2.txt}
\subsection{Połączenie między routerami}
\section{Konfiguracja ipv6}
\subsection{Konfiguracja routera R1}
\subsection{Konfiguracja routera R2}
By skonfigurować R2 pod użycie ipv6 wchodzimy w tryb konfiguracji i ustawiamy ipv6 adres.
\inputminted[label=Output, firstline=272, lastline=283]{text}{R2.txt}
R2 znajduje się pod adresem 2001:DB8:0:1::1.
\inputminted[label=Output, firstline=309, lastline=323]{text}{R2.txt}
Usuwamy ustawiony wcześniej ipv4 adres, dodajemy unicast routing.
\inputminted[label=Output, firstline=324, lastline=329]{text}{R2.txt}
\inputminted[label=Output, firstline=376, lastline=382]{text}{R2.txt}
R2 jest gotowy do łączenia przez Ethernet 0/0 korzystając z ipv6.
\inputminted[label=Output, firstline=383, lastline=402]{text}{R2.txt}
\subsection{Połączenie między routerami}
\end{document}
